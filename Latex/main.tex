\documentclass[12pt]{article}
% pre\'ambulo
\usepackage{lmodern}
\usepackage[T1]{fontenc}
\usepackage[spanish,activeacute]{babel}
\usepackage{mathtools}
\usepackage{url}
\usepackage{tikz}
\usepackage{graphicx}
\graphicspath{ {images/} }
\usetikzlibrary{arrows,positioning}
\tikzset{
    %Define standard arrow tip
    >=stealth',
    %Define style for boxes
    punkt/.style={
           rectangle,
           rounded corners,
           draw=black, very thick,
           text width=5em,
           minimum height=4em,
           text centered},
    % Define arrow style
    pil/.style={
           ->,
           thick,
           shorten <=2pt,
           shorten >=2pt,}
}
\usepackage[style=numeric-comp,backend=bibtex]{biblatex}
\bibliography{refs.bib}

\title{Modelo de Actores como Modelo de Agentes}
\author{Sergio Rodr\'iguez Calvo}
\date{Junio de 2017}


\begin{document}
    \maketitle
    \thispagestyle{empty}
    \begin{center}
      Departamento de Ciencias de la Computaci\'on e
      Inteligencia Artificial \\
      Universidad de Sevilla
      \end{center}
  % cuerpo del documento
  % abstract
  \bf{Abstract. }\rm
    \emph{En el presente trabajo se muestran dos modelos que tienen similitudes
    pero que aplican en entornos distintos, estos son el modelo de agentes y
    el modelo de actores. El primero de ellos, el modelo de agentes, consiste en
    un conjunto de componentes, iguales o diferentes entre ellos, que actuan de forma
    racional y, en conjunto, tienen un comportamiento complejo. El segundo de ellos,
    el modelo de actores, aplica en el mundo de la informaci'on, es decir, se ejecutan
    en computadores y se comunican entre ellos realizando cada uno peque'nas tareas,
    permitiendo que los sistemas sean robustos, escalables y tolerantes a fallos.}
\section{Introducci'on}
En la naturaleza a menudo se observan comportamientos complejos y, en muchos casos,
pueden llegar a considerarse ciertamente inteligentes. Al realizar un estudio en profundidad
de los mismos se aprecia como el sistema se compone de seres, individuos o agentes cuya operativa
es muy simple. Un ejemplo de ello lo tenemos en una colonia de hormigas, el cual es un
sistema complejo que resuelve problemas de envergadura, sin embargo, los individuos de
una colonia de hormigas son insectos que tienen labores espec'ificas y que aplican en
todos los casos sea cuales sean las circustancias. Es un sistema que cuenta con distintos
tipos de agentes, hormigas en este caso, y cada una de ellas tiene un enfoque o una labor diferente.

No se puede decir que una hormiga sea inteligente, o al menos, no tiene un grado elevado de la misma.
En esencia, consiste en unos sentidos muy b'asicos mediante los cuales percibe el entorno y
de su an'alisis tiene lugar una acci'on, actuaci'on o respuesta, racional.

Este paradigma ha sido observado por las personas y de su estudio ha surgido un paradigma o
enfoque para resolver diferentes problemas en diferentes 'areas. Esto se conoce como Agentes
Inteligentes, aunque puede encontrarse con otro nombre como Agentes Racionales dada la dificultad
que supone definir el concepto inteligencia.

Una de las virtudes que presenta un enfoque de este tipo es la posibilidad de ser masivamente
paralelo. Hoy d'ia, resultado de a'nos de evoluci'on y sofisticaci'on de los sistemas de la informaci'on,
se necesita construir sistemas que tengan gran capacidad, que sean escalables y el'asticos
en funci'on de la carga que soporte el sistema en cada momento, y que tambi'en sean tolerantes
a fallos, siendo incluso capaces de saber reponerse en caso de que se produzcan errores o fallos.

De todo ello, ha surgido una corriente que busca construir sistemas en pequeños componentes. E incluso,
que esos pequeños componentes sean descritos como pequeños agentes de la informaci'on con tareas simples,
que en conjunto componen un sistema que realiza tareas con muy buen rendimiento. Se le conoce como modelo de
actores, y consiste en pequeños trozos de c'odigo software que ante la llegada de un mensaje realizan
una tarea, y finalmente, devuelven un resultado.

Todo esto no es nuevo, si no que supone aprovechar un modelo que se creo en la decada de los 80, y
que es ampliamente aplicado desde entonces en el mundo de las comunicaciones. Consiste en un
modelo matem'atico que resuleve muy bien problemas como los descritos previamente.

\section{Modelo de Agentes}
\label{sec:modelo de agentes}
Lorem ipsum.
\subsection{Qu'e es un agente}
\label{sub:que es un agente}
Lorem ipsum.
\subsection{Capacidad racional en agentes}
\label{sub:capacidad racional en agentes}
Lorem ipsum.
\section{Modelo de Actores}
\label{sec:modelos de actores}
En la construcci'on de sistemas de informaci'on se plantea un nuevo mundo, un mundo en el cual
las aplicaciones tengan una alta capacidad de respuesta. Se necesita mantener la atenci'on y el
inter'es de los usuarios que acceden a ellos.

Las diferencias entre estos dos mundos, el mundo que conocemos hasta ahora y el que se plantea
desde hace ya unos a'nos, cuenta con las siguientes diferencias:

\begin{itemize}
	\item Una 'unica computadora frente a un cluster de computadores.
	\item Un 'unico n'ucleo de procesamiento frente a m'ultiples de ellos.
	\item Alto coste de las memorias RAM frente a memorias RAM de bajo coste.
	\item Alto coste del almacenamiento f'isico frente a bajo coste del mismo.
    \item Redes lentas frente a redes de comunicaciones ultra r'apidas.
    \item Baja concurrencia de usuarios frente a alta concurrencia.
    \item Conjunto peque'no de datos frente a grandes conjuntos de datos.
    \item Latencia en segundos frente a latencia en milisegundos.
\end{itemize}

El modelo de actores fue creado en 1973 por Carl Hewitt y consiste en un modelo
matem'atico de computaci'on concurrente cuya primitiva universal es el actor. Un actor es
una proci'on de c'odigo software que puede ejecutarse m'ultiples veces, e incluso,
de forma paralela.

Esta idea puede ser llevada m'as all'a gracias a una implementaci'on adecuada de un toolkit
que permita construir sistemas distribuidos siguiendo un modelo de actores y que a su vez
sea fiel al manifiesto reactivo. No s'olo es una mejora, si no que se hace necesario dado
el entorno en el que se desarrolla y se pretende ejecutar, un computador.

\subsection{Qu'e es un actor}
\label{sub:que es un actor}
Un actor es la primitiva universal de concurrencia dentro del modelo de actores. Esto es,
un bloque de c'odigo por el que pasa un hilo de ejecuci'on cada vez siempre y cuando
tenga un mensaje pendiente.

Su topolog'ia consta de un buz'on o cola de mensajes, y el propio actor. Un actor consulta su
buz'on para comprobar si existe en el mensajes pendientes y en funci'on de un dispatcher, elemento
que da paso al actor para que se ejecute, extrae el mensaje y ejecuta la acci'on para la que
esta programado. Una vez finaliza su ejecuci'on, y dependiendo de la operativa, devolver'a una
respuesta, o realizar'a otras acciones tales como reenviar el mensaje, crear otro actor
para delegar en 'el, entre otras.

Un actor no debe guardar estados, ya que en caso de que un mensaje por alg'un motivo
volviera a encolarse en el buz'on, el actor ha podido procesar otros mensajes previamente
que han podido dejar el estado de forma que no permita alcanzar el resultado esperado, teniendo
por tanto un efecto no deseado.

\subsection{Sistemas distribuidos}
\label{sub:sistemas distribuidos}
Un sistema distribuido se define como un conjunto de computadores conectados entre s'i. Esto
es la soluci'on para poder construir sistemas reactivos.

En un primer momento, los procesadores, construidos con la conocida como tecnolog'a del
silicio, iban doblando su potencia cada 18 meses. Esto es lo que se conoce como Ley de Moore.
Por ello, y gracias tambi'en al avaratamiento de la tecnolog'ia y al aumento de la capacidad de
almacenamiento f'isico y de memoria RAM, no era necesario una alternativa a la construci'on de sistemas
monol'iticos, es decir, sistemas desplegados en un 'unico computador y que contara con
todos los servicios.

Tampoco la sociedad utilizaba internet y los sistemas web para tantas tareas como ahora, por lo
que con un 'unico sistema monol'itico y un computador con suficiente recursos (potencia y memoria)
era suficiente.

A lo largo de los años se ha ido alcanzando los l'imites de la tecnolog'ia y nuevos paradigmas o
enfoques han ido surgiendo. Por ejemplo, casi simult'aneamente a los sistemas distribuidos se
hizo necesario duplicar el numero de procesadores. Es decir, contar con varios n'ucleos de
procesamiento en un mismo procesador.

Los sistemas distribuidos, por tanto, suponen llevar m'as all'a el enfoque aplicado con el enfoque
multin'ucleo, y aplicarlo tambi'en a nivel de computadores aprovechando las redes de comunicaciones
existentes entre ellos.

\subsection{Manifiesto Reactivo}
\label{sub:manifiesto reactivo}
El manifiesto reactivo es un documento en el que se recoge los principios que debe
tener en cuenta todo sistema que pretenda ser reactivo. Esto es, ser flexible, con bajo
acoplamiento y escalables.

Estos principios son los siguientes:

\begin{itemize}
	\item Responsividad: Los sistemas responden de forma adecuada. Se requiere de tiempos
    de respuesta r'apidos y consistentes.
	\item Resilencia: Los sistemas permanecen responsivos incluso en situaci'on de fallo.
    Entendiendo esto como alta disponibilidad. Esto ocurre gracias a la replicaci'on, contenci'on,
    aislamiento y la delegaci'on.
	\item Elasticidad: Los sistemas permanecen responsivos incluso ante variaciones de alta
    carga de trabajo.
	\item Orientaci'on a mensajes: Los sistemas reactivos intercambian mensaje. No mantienen estados,
    si no que el estado es el mensaje. Puede verse como un sistema orientado a eventos u 'ordenes.
\end{itemize}

\subsection{Rol de los actores en sistemas distribuidos (y reactivos)}
\label{sub:rol de los actores en sistemas distribuidos y reactivos}
Lorem ipsum.
\subsection{Capacidad racional en actores}
\label{sub:capacidad racional en actores}
Lorem ipsum.
\section{Ventajas e Inconvenientes}
\label{sec:ventajas e inconvenientes}
Lorem ipsum.
\section{Conclusi'on}
Lorem ispum.
\section{Referencias}
Lorem ispum.
\end{document}
