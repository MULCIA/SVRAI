\documentclass[12pt]{article}
% pre\'ambulo
\usepackage{lmodern}
\usepackage[T1]{fontenc}
\usepackage[spanish,activeacute]{babel}
\usepackage{mathtools}
\usepackage{url}
\usepackage{tikz}
\usepackage{graphicx}
\graphicspath{ {images/} }
\usetikzlibrary{arrows,positioning}
\tikzset{
    %Define standard arrow tip
    >=stealth',
    %Define style for boxes
    punkt/.style={
           rectangle,
           rounded corners,
           draw=black, very thick,
           text width=5em,
           minimum height=4em,
           text centered},
    % Define arrow style
    pil/.style={
           ->,
           thick,
           shorten <=2pt,
           shorten >=2pt,}
}
\usepackage[style=numeric-comp,backend=bibtex]{biblatex}
\bibliography{refs.bib}

\title{Modelo de Actores como Modelo de Agentes}
\author{Sergio Rodr\'iguez Calvo}
\date{Junio de 2017}


\begin{document}
    \maketitle
    \thispagestyle{empty}
    \begin{center}
      Departamento de Ciencias de la Computaci\'on e
      Inteligencia Artificial \\
      Universidad de Sevilla
      \end{center}
  % cuerpo del documento
  % abstract
  \bf{Abstract. }\rm
    \emph{En el presente trabajo se presentan el modelo de acotres y el modelo de agentes
    realizando una comparaci'on y un an'alisis de similitudes.}
\section{Introducci'on}
This is where you will write your content.
\end{document}
